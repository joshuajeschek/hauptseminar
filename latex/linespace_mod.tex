%
% Eine Änderung des Zeilenabstands wirkt sich nur auf Mengentext aus. In
% Fußnoten und Gleitumgebungen kommt der einfache Zeilenabstand zur Anwendung.
% Der folgende TeX-Code aktiviert den erweiterten Zeilenabstand für diese
% Fälle.
%


%
% Erweiterter Zeilenabstand in Gleitumgebungen
%
\makeatletter
\let\@xfloat=\latex@xfloat
\makeatother


%
% Erweiterter Zeilenabstand in Fußnoten
%
\makeatletter
\long\def\@footnotetext#1{%
  \insert\footins{%
    \reset@font\footnotesize
    \interlinepenalty\interfootnotelinepenalty
    \splittopskip\footnotesep
    \splitmaxdepth \dp\strutbox \floatingpenalty \@MM
    \hsize\columnwidth
    \@parboxrestore
    \protected@edef\@currentlabel{%
      \csname p@footnote\endcsname\@thefnmark
    }%
    \color@begingroup
    \@makefntext{%
      \rule\z@\footnotesep\ignorespaces#1\@finalstrut\strutbox}%
    \color@endgroup}}
\makeatother


%
% Erweiterter Zeilenabstand in Fußnote von minipages
%
\makeatletter
\long\def\@mpfootnotetext#1{%
  \global\setbox\@mpfootins\vbox{%
    \unvbox \@mpfootins
    \reset@font\footnotesize
    \hsize\columnwidth
    \@parboxrestore
    \protected@edef\@currentlabel{%
      \csname p@mpfootnote\endcsname\@thefnmark}%
    \color@begingroup
    \@makefntext{%
      \rule\z@\footnotesep\ignorespaces#1\@finalstrut\strutbox}%
    \color@endgroup}}
\makeatother
