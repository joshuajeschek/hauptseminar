\chapter{Fazit}

Im Rahmen dieser Arbeit wurden die Sicherheit und die Usability verschiedener \acl{2FA}-Methoden untersucht. Die Ergebnisse verdeutlichen, dass jede der betrachteten Methoden - von \acp{OTP}, \acp{TOTP}, vor-generierten Codes, Push-Benach\-richtigungen bis hin zu \ac{U2F}-Security-Keys - sowohl Vor- als auch Nachteile aufweist.

Obwohl \ac{U2F}-Security-Keys die höchste Sicherheit bieten, werden sie in der Praxis aufgrund ihrer höheren Einrichtungs­komplexität und der Notwendigkeit eines zusätzlichen physischen Geräts weniger genutzt. Im Gegensatz dazu erfreuen sich \acp{OTP} über SMS und E-Mail trotz ihrer Sicherheitsrisiken großer Beliebtheit, da sie bereits bekannt sind und ohne zusätzliche Hardware einfach verwendet werden können.

Die Untersuchung hat zudem gezeigt, dass die Verbreitung von \ac{2FA} im privaten Bereich trotz der bekannten Sicherheitsvorteile weiterhin gering ist. Dies liegt häufig an einer Mischung aus mangelndem Bewusstsein für die Bedrohungen und der Wahrnehmung, dass der zusätzliche Aufwand der Nutzung von \ac{2FA} keinen signifikanten Sicherheitsvorteil bietet. Um die Akzeptanz von \ac{2FA} zu steigern, sollten sowohl die Usability der Verfahren verbessert als auch die Vorteile klarer kommuniziert werden.

Zukünftige Forschung sollte sich darauf konzentrieren, die Usability von \ac{2FA}-Methoden weiter zu optimieren, insbesondere im Hinblick auf die Einstiegshürden bei sicheren Methoden wie \ac{U2F}-Security-Keys. Zudem könnten Anreize und bessere Aufklärungskampagnen dazu beitragen, die Akzeptanz in der breiten Bevölkerung zu fördern. So könnte \ac{2FA} zu einem festen Bestandteil des Alltags vieler Nutzerinnen werden und mehr Accounts effektiv schützen.
