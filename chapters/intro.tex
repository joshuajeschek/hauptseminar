\newcommand{\gfn}{\footnote{In dieser Arbeit wird das generische Femininum genutzt. Die in dieser Arbeit verwendeten Personenbezeichnungen beziehen sich - sofern nicht anders kenntlich gemacht - auf alle Geschlechter.}}

\chapter{Einleitung}

Passwortlecks sind ein immer wiederkehrendes Thema in den Medien. Ob durch Angriffe auf große Unternehmen \parencite{mccandlessWorldsBiggest2024}, oder zunehmend raffinierte Phishing-Attacken \parencite{colnagoItsNot2018}, die Veröffentlichung von E-Mail-Adressen und Passwörtern stellt ein erhebliches Risiko für Nutzerinnen\gfn dar. Schlechte Passwortgewohnheiten, wie die Wiederverwendung von Passwörtern oder die Nutzung leicht zu erratender Passwörter, verstärken dieses Risiko zusätzlich \parencite{colnagoItsNot2018}.

\ac{2FA} bietet jedoch einen wirksamen Schutz gegen Cyberangriffe. Durch die Verwendung eines zweiten, individuellen Sicherheitsfaktors können selbst Accounts mit schwachen oder mehrfach genutzten Passwörtern im Falle eines Datenlecks besser geschützt werden.

Trotz der zunehmenden Medienpräsenz von Datenlecks sind die Nutzungsraten von 2FA weiterhin gering \parencite{petsasTwofactorAuthentication2015, ackermanImpedimentsAdoption2020}.  Diese Diskrepanz wirft die Frage auf, warum viele Nutzerinnen \ac{2FA} nicht anwenden und wie mehr Menschen dazu bewegt werden können, diese Sicherheitsmaßnahme zu nutzen.


Die vorliegende Arbeit untersucht zunächst verschiedene \ac{2FA}-Methoden:, darunter \aclp{OTP}, \aclp{TOTP}, vor-generierte Codes, Push-Benachrichtigungen und \ac{U2F} Security Keys. Dabei werden sowohl Sicherheitsaspekte als auch die Benutzerfreundlichkeit der einzelnen Methoden beleuchtet. In Kapitel \ref{sec:adoption} wird analysiert, welche Hindernisse die Nutzung von \ac{2FA} erschweren und welche Maßnahmen zur Erhöhung der Nutzungsrate beitragen können.
