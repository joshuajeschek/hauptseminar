\chapter{Zwei-Faktor-Authentifizierungsmethoden}
Die \ac{2FA} ist ein Sicherheitsverfahren, welches zwei unterschiedliche Komponenten zur Verifizierung der Identität von Nutzerinnen erfordert \parencite{decristofaroComparativeUsability2014}. Verschiedene Publikationen definieren dazu drei Kategorien \parencite{decristofaroComparativeUsability2014, yuEfficientGeneric2014, singhMultifactorAuthentication2017}:
\begin{itemize}
  \item \textbf{Wissen}: Etwas, was die Benutzerin weiß (z.B. Passwort, PIN)
  \item \textbf{Besitz}: Etwas, was die Benutzerin besitzt (z.B. Smartphone, Physisches Token)
  \item \textbf{Inhärenz}: Etwas, was die Benutzerin ist (z.B. Fingerabdruck, Gesichtserkennung)
\end{itemize}
Sobald Elemente aus verschiedenen Kategorien genutzt werden, handelt es sich um Multi-Faktor-Authentifizierung. Am meisten verbreitet ist \ac{2FA}, wobei Faktoren aus zwei der Kategorien abgefragt werden \parencite{decristofaroComparativeUsability2014}. Am Häufigsten wird \ac{2FA} mit einen Passwort und einem weiteren Faktor umgesetzt \parencite{decristofaroComparativeUsability2014}.

\pskip
\ac{2FA} stellt eine Verbesserung der Sicherheit gegenüber traditionellen Methoden, die nur einen Faktor verwenden (z.B. Passwörter), da es schwieriger ist, beide Faktoren zu kompromittieren \parencite{dasguptaMultiFactorAuthentication2017}.

Des Weiteren werden gängige Angriffsmethoden wie Phishing oder Keylogging erschwert. Angreiferinnen können selbst wenn sie das Passwort wissen, nicht auf die Konten der Benutzerinnen zugreifen \parencite{dasguptaMultiFactorAuthentication2017}. Auch wenn Passwörter im Rahmen eines Daten-Leaks an die Öffentlichkeit geraten, ermöglicht dies keinen sofortigen Zugriff und gibt den Betroffenen Zeit, ihr Passwort anzupassen.

\pskip
In vielen Bereichen, wie beispielsweise dem Online-Banking und Regierungsbehörden, wird \ac{2FA} eingesetzt und ist Pflicht. Dennoch ist die Entwicklung von \ac{2FA}-Methoden noch nicht abgeschlossen und birgt einige Herausforderungen.

Einerseits kann die Einführung eines zusätzlichen Authentifizierungsschrittes zu Lasten der Benutzerinnenerfahrung gehen \parencite{decristofaroComparativeUsability2014}. Andererseits werden durch zusätzliche Faktoren auch zusätzliche Implementierungskosten hervorgerufen, sei dies durch Hardware-Kosten (z.B. für Security-Tokens), technische Implementierung, oder Support \parencite{alsaleemMultiFactorAuthentication2021}.

\pskip
Im folgenden werden fünf gängige \ac{2FA}-Methoden vorgestellt. Dabei wird sowohl auf deren Funktionsweise, als auch auf Usability-Probleme und Sicherheitserwägungen eingegangen.

\section{SMS}
Sim Swapping
SS7 Attacks
\section{TOTP}
\section{Vor-Generierte Codes}
\section{Push Benachrichtigungen}
\section{U2F Security-Tokens}
